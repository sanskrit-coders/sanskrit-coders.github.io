% Template - https://sanskrit.uohyd.ac.in/18WSC/Style_files/CS_and_DH.tex
\providecommand{\tightlist}{%
  \setlength{\itemsep}{0pt}\setlength{\parskip}{0pt}}

\documentclass[11pt]{article}
\usepackage{scl}
\usepackage{times}
\usepackage{url}
\usepackage{latexsym}
\usepackage{lineno}

\usepackage{fontspec, xunicode, xltxtra}
\newfontfamily\skt[Script=Devanagari]{Sanskrit 2003}
\setmonofont{Sanskrit 2003}


\title{A general Dictionary distribution framework}

\author{
  Vishvas Vasuki \\
  Dyugaṅgā, Beṅgaḷūru \\
  {\tt https://sanskrit.github.io/groups/dyuganga/}
\\}

\date{}

\begin{document}
\maketitle
%\linenumbers
\begin{abstract}
Dictionaries are the constant companions of scholars and students, and there are lot of them (some being continuously updated). They are often consumed offline, on a variety of devices including one's phone. We describe a general dictionary distribution framework which serves hundreds of such dictionaries to a sizable user community.
\end{abstract}

\section{Motivation}
Today's scholars and students frequently want to refer dictionaries on electronic devices ranging from desktop computers to phones and voice-AI devices like Alexa. Dictionaries of interest may be large in number and size. For example, there are hundreds of dictionaries of interest to Indological scholars accross a variety of languages and topics. Furthermore, some of these dictionaries are constantly being updated. This being the case, it becomes problematic to curate, distribute and update such dictionaries for the convenience of both creators and users. We present an open source and extensible system to address this problem. Some of its features are listed below.

\begin{itemize}
\tightlist
\item \textbf{For creators}
\item \textbf{For readers}
\end{itemize}



\begin{figure}[h]
\caption{Screenshot from GoldenDict}
\centering
\includegraphics[width=1.0\textwidth]{images/kalpAntaram-screenshot}
\end{figure}


\section{Future work}

\bibliographystyle{acl}
\bibliography{dict-distribution}
\end{document}